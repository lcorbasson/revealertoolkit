
\documentclass[a4paper,11pt,oneside]{report}
\usepackage[utf8]{inputenc}
\usepackage[pdftex]{color,graphicx} 
\usepackage[pdftex,colorlinks]{hyperref} 
\usepackage{xspace}
\usepackage{longtable}
\usepackage{multirow}
\usepackage{pdflscape}
\usepackage{subfigure}
\usepackage{pdfpages}


\title{Revealer Toolkit -- User Guide}
\author{Jose Navarro}


\begin{document}

\maketitle

\tableofcontents

\addtolength{\parskip}{0.5 \baselineskip}

\chapter{License}

\section{License of this document}

    Copyright (C) 2009 Jose Navarro a.k.a. Dervitx
    
    Permission is granted to copy, distribute and/or modify this document
    under the terms of the GNU Free Documentation License, Version 1.3
    or any later version published by the Free Software Foundation;
    with no Invariant Sections, no Front-Cover Texts, and no Back-Cover Texts.
    
    A copy of the license is included in the section entitled "GNU
    Free Documentation License".
    
    More information at: http://www.gnu.org/licenses/fdl.html

\section{License of the Revealer Toolkit}

	Copyright (C) 2008 Jose Navarro a.k.a. Dervitx

    This program is free software; you can redistribute it and/or modify
    it under the terms of the GNU General Public License as published by
    the Free Software Foundation; either version 2 of the License, or
    any later version.

    This program is distributed in the hope that it will be useful,
    but WITHOUT ANY WARRANTY; without even the implied warranty of
    MERCHANTABILITY or FITNESS FOR A PARTICULAR PURPOSE.  See the
    GNU General Public License for more details.

    You should have received a copy of the GNU General Public License along
    with this program; if not, write to the Free Software Foundation, Inc.,
    51 Franklin Street, Fifth Floor, Boston, MA 02110-1301 USA.
    
     For more information, please visit
     http://www.gnu.org/licenses/old--licenses/gpl--2.0.txt


\chapter{Acknowledgments}

\begin{itemize}
     \item INCIDE (Investigacion Digital S.L., www.incide.es)
       where developers and testers work
     \item Manu Ginés aka xkulio 
       creator of the original Chanchullos Revealer
     \item Generalitat de Catalunya
       for partial funding of the project
\end{itemize}



\chapter{Introduction}

the Revealer Toolkit is a framework and simple scripts for computer forensics. It uses Brian Carrier's The Sleuth Kit as the backbone, as well as other free tools.

The aim of the Revealer Toolkit is to automate rutinary tasks and to manage sources and results from another perspective than the usual forensic frameworks.

RVT is developed and actively tested by computer forensic investigators working at INCIDE, spanish company sited at the beautiful city of Barcelona (see www.incide.es for more details)

project state: we are using RVT actually in our production servers, although do not expect an easy and clean software yet or a quick installation. For any questions, help or comments, please, drop an email.

You can find additional information, packages and all the source code at http://code.google.com/p/revealertoolkit



% -----------------------------------
% Installation
% -----------------------------------


\chapter{Installation} \label{sec:installation}

This is and example installation of the Revealer Toolkit Shell on a Debian \emph{lenny} 5.00. Also covers the creation of a valid folder structure and a sample image.

\section{Operating system}

Download last Debian stable version. In this example, Debian 5.00 \emph{lenny} is used.

Some additional packages are required to be installed with \emph{apt-get}:

\begin{verbatim}
apt-get install openssh-server
apt-get install unzip
apt-get install sudo vim
apt-get install sleuthkit binutils
apt-get install dosfstools

groupadd -g 1010 forensics
addgroup analyst forensics

mkdir /media/morgue
chgrp forensics morgue
chmod g+sw morgue

mkdir /media/datos
chgrp forensics datos
chmod g+sw datos

chmod a+r /dev/loop*
and add the last line to the  \emph{/etc/init.d/bootmisc.sh}  file (before the last \emph{:})
\end{verbatim}


as root, add this line to the sudoers file (with \emph{visudo} command):

\begin{verbatim}
Defaults:%forensics !authenticate

%forensics ALL=(root) /bin/mount, (root) /bin/umount, (root) /sbin/losetup
\end{verbatim}


\section{Software}

\begin{verbatim}
wget revealertoolkit.googlecode.com/files/RVT_v0.1.zip
unzip RVT_v0.1.zip
chmod a+x RVT/RVT.pl
\end{verbatim}

as root:

\begin{verbatim}
ln -s /home/analyst/RVT/RVT.pl /usr/bin/RVT.pl
\end{verbatim}


\section{Folder structure}

as user:

\begin{verbatim}
mkdir /media/morgue/imagenes
mkdir /media/datos/imagenes
\end{verbatim}

\begin{verbatim}
mkdir /media/morgue/imagenes/100101-ghost
mkdir /media/morgue/100101-ghost
mkdir /media/morgue/100101-ghost/100101-01-1
mkdir /media/morgue/100101-ghost/100101-01-1/mnt
mkdir /media/morgue/100101-ghost/100101-01-1/output
\end{verbatim}



\section{Sample image creation}

as user:

\begin{verbatim}
cd /media/morgue/imagenes/100101-ghost/
dd if=/dev/zero bs=1024 count=10240 > 100101-01-1.dd
\end{verbatim}

as root:

\begin{verbatim}
fdisk /media/morgue/imagenes/100101-ghost/100101-01-1.dd

in fdisk:

x (additional functions)

c (cylinders)
1024
s (sectors)
10
h (heads)
2

r (main menu)
n (new partition)
p 
1
1   (first cylinder)
512 (half of the disk)
n (new partition)
p 
2
513 (half of the disk)
1024 (last cylinder)
w (save and exit)
\end{verbatim}

and now, let's put some info inside, as root:

\begin{verbatim}
echo "losetup assign 100101" | RVT.pl -b
echo "losetup list" | RVT.pl -b  
   (the last command reveals the loop devices created)
mkfs.vfat /dev/loop0
mkfs.vfat -F 32 /dev/loop1   

mkdir /media/aux1
mount /dev/loop0 /media/aux1
cp /home/analyst/RVT/RVT.pl /media/aux1
umount /media/aux1
mount /dev/loop1 /media/aux1
cp /home/analyst/RVT/RVT.pl /media/aux1
sync
rm /media/aux1/RVT.pl
echo "my email address is myemail@revealertoolkit.com" > /media/aux1/textfile.txt
umount /media/aux1

echo "losetup delete 100101" | RVT.pl -b
\end{verbatim}


\section{First test}

Just for testing, some first commands can be executed. As analyst, execute the RVT Shell:

\begin{verbatim}
\$ RVT.pl 
\end{verbatim}

After the preliminar scanning, RVT Shell will offer you a prompt:

\begin{verbatim}
RVT >
\end{verbatim}

Now, execute these commands:

\begin{verbatim}
set level 100101-01-1
script strings generate
script timelines generate
script webmail detection
script software detection
script search quickcount emails
script search quickcount accounts
quit
\end{verbatim}

You can check that the directory \emph{/media/morgue/100101-ghost/100101-01-1/output} has been populated with results.

The same can be achieved creating a file with the commands and piping it to the RVT Shell using the \emph{-b} argument:

\begin{verbatim}
cat preforensics.rvt | RVT -b
\end{verbatim}






% -----------------------------------
% Folder structure
% -----------------------------------

\chapter{Folder structure}

The morgue stores disk images and results of forensic analysis. Each morgue must have a strict folder structure. 

\section{Morgues}

The Revealer Toolkit can handle more that one morgue. By default, two morgues are defined in RVT, mounted at \emph{/media/morgue} and \emph{/media/datos}. 

These morgues can be managed modifying the RVT Shell internal code. See the RVT Developer Guide for more info.


\section{Cases}

Each forensic case is determined by a \emph{case number} and a \emph{case codename}, separated by a dash. For example, the example case created in the Chapter \ref{sec:installation} is noted as \emph{100101-ghost}.

Each case has a folder assigned in the morgue, under the folder \emph{images}, where the disk images are stored.

Also, each case has a folder in the morgue, where all the forensic results are stored.

The folder structure, at the case level, for the example installation shown in chapter \ref{sec:installation}, will be:

\begin{verbatim}
/media/morgue/100101-ghost

/media/morgue/imagenes/100101-ghost
/media/morgue/imagenes/100101-ghost/100101-01-1.dd
\end{verbatim}

where \emph{100101-01-1.dd} is the dd image of a disk.


\section{Devices, disks and partitions}

Under the Revealer Toolkit, information sources are organized with:

\begin{itemize}
\item Devices: each case has a number of devices: computers, cell phones, digital cameras, ...   They are numbered sequencially from 01 to 99. 

A device is noted as \emph{casenumber-devicenumber}, for example, \emph{100101-01} for the case 100101 and device 01.

\item Disks: each device has a number of disks: hard disks, CD's, memory cards, ... They are numbered sequencially from 1 to 9

A disk is noted as \emph{case-device-disknumber}, for example, \emph{100101-01-1}, for the disk 1

\item Partitions:  each disk can have several partitions, numbered from 01 to 99. The numeration used by the Sleuthkit command \emph{mmls} is used.

A partition is noted as \emph{case-device-disk-ppartition}, for example, 100101-01-1-p02 for partition 02.
\end{itemize}


Under each case folder, a folder must exist for every disk to be analyzed. 

This folder structure, at a disk level, for the example installation shown in chapter \ref{sec:installation}, will be:

\begin{verbatim}
/media/morgue/100101-ghost
/media/morgue/100101-ghost/100101-01-1

/media/morgue/imagenes/100101-ghost
/media/morgue/imagenes/100101-ghost/100101-01-1.dd
\end{verbatim}


\section{Forensic results}

The Revealer Toolkit Shell manages and executes \emph{script modules}, which performs forensic operations on the disk images and disk information. The results of these script modules are stored in the corresponding disk folder, under a folder named \emph{output}.

The folder structure and file content stored here depends on each script module. See chapter \ref{sec:modules} for further information.

Furthermore, under the disk folder other folder exists, named \emph{mnt}, that contains the mounting points for the image partitions. 

Then, the complete folder structure for the example shown in chapter \ref{sec:installation} will be:

\begin{verbatim}
/media/morgue/100101-ghost
/media/morgue/100101-ghost/100101-01-1
/media/morgue/100101-ghost/100101-01-1/mnt
/media/morgue/100101-ghost/100101-01-1/output

/media/morgue/imagenes/100101-ghost
/media/morgue/imagenes/100101-ghost/100101-01-1.dd
\end{verbatim}


% -----------------------------------
% Command guide
% -----------------------------------


\chapter{Command guide}

The Revealer Toolkit Shell provide several commands used to (a) manage your forensic images and (b) execute forensic operations over them.

RVT Shell is a Perl script that, when executed, performs a scan of the morgue and of all the images stored into it. After that, a prompt is shown and commands can be introduced.



\section{General interaction}

\begin{itemize}
\item to execute one command, type it at the prompt, add the corresponding arguments, and type RETURN. Some commands return information on the screen, some, write information at the \emph{output} folder of the corresponding disk at the morgue. Other, do both.

\item type one command followed by \emph{?} to obtain help about it

\item type the TAB key to obtain a list of available commands

\end{itemize}

\begin{verbatim}
Welcome to Revealer Tools Shell (v0.1.0):

 RVT  > 
        case
        images
        info
        losetup
        mount
        script
        set
        test
 RVT  > 
\end{verbatim}


\section{Command: \emph{case}}

Case management.

\subsection{\emph{case list}}

Gives a list of the cases stored in the morgue.

\begin{verbatim}
 RVT  > case list
Cases in the morgue: 
        100101 'ghost':
                100101-01-1

 \end{verbatim}



\section{Command: \emph{images}}

Image management.



\subsection{\emph{images list}}

Gives a list of the disk images stored in the morgue.

\begin{verbatim}
 RVT  > images list
Images in the morgue: 
        100101 'ghost':
                100101-01-1.dd

\end{verbatim}



\subsection{\emph{images partition info}}

Gives information about a partition.

\begin{verbatim}
 RVT  > images partition info 100101-01-1-p03                                 

Info for partition  - 100101-01-1-p03:

Filesystem:     FAT12
Cluster size:   2048
Sector size:    512
Offset:         0000010240 sectors ( 5242880 bytes )
\end{verbatim}

\subsection{\emph{images partition table}}

Gives the partition table of an image.

\begin{verbatim}
 RVT  > images partition table 100101-01-1

        03:     6 MB    Linux (0x83)
        02:     5 MB    Linux (0x83)
\end{verbatim}


\subsection{\emph{images scanall}}

RVT scans the morgue. No output given. 

\section{Command: \emph{info}}

Manages RVT Shell configuration information.

\subsection{\emph{info list}}

Gives RVT Shell configuration information.

\begin{verbatim}
 RVT  > info list

        List of morgues:
                /media/morgue 
                /media/datos 


        List of morgues of images:
                /media/morgue/imagenes 
                /media/datos/imagenes 
\end{verbatim}


\section{Command: \emph{losetup}}

Loop device management.

\subsection{\emph{losetup assign}}

Assigns a loop device to each partition of a case.

\begin{verbatim}
 RVT  > losetup assign 100101

sudo losetup -f /media/morgue/imagenes/100101-ghost/100101-01-1.dd -o 5242880

sudo losetup -f /media/morgue/imagenes/100101-ghost/100101-01-1.dd -o 5120
\end{verbatim}


\subsection{\emph{losetup delete}}

Delete loop devices assigned to a case

\begin{verbatim}
 RVT  > losetup delete 100101
\end{verbatim}


\subsection{\emph{losetup list}}

List all loop devices assigned.

\begin{verbatim}
 RVT  > losetup list                                                          
Loop devices: 
        loop1   100101-01-1.dd  5120
        loop0   100101-01-1.dd  5242880
\end{verbatim}


\subsection{\emph{losetup recheck}}

Updates loop device information from the operating system.


\section{Command: \emph{mount}}

Mount points management.

\subsection{\emph{mount assign}}

Mounts the partitions of a case at the path \emph{morgue/100xxx-case/100xxx-device-disk/mnt/p0N}, where N is the partition number.

\begin{verbatim}
 RVT  > mount assign 100101

sudo mount /media/morgue/imagenes/100101-ghost/100101-01-1.dd /media/morgue/100101-ghost/100101-01-1/mnt/p03 -o ro,loop,iocharset=utf8,offset=5242880,umask=007,gid=1001

sudo mount /media/morgue/imagenes/100101-ghost/100101-01-1.dd /media/morgue/100101-ghost/100101-01-1/mnt/p02 -o ro,loop,iocharset=utf8,offset=5120,umask=007,gid=1001
\end{verbatim}


\subsection{\emph{mount delete}}

Umounts partitions of a case.

\begin{verbatim}
 RVT  > mount delete 100101
\end{verbatim}


\subsection{\emph{mount list}}

List all mounted points.

\begin{verbatim}
 RVT  > mount list
Mounted partitions: 
        100101-01-1.dd  loop=/dev/loop0 offset=5242880
        100101-01-1.dd  loop=/dev/loop1 offset=5120
\end{verbatim}


\subsection{\emph{mount recheck}}

Updates mount points information from the operating system.



\section{Command: \emph{cluster}}

Operations on clusters.

\subsection{\emph{cluster allocationstatus}}

Prints cluster allocation status

\begin{verbatim}
 RVT  > cluster allocationstatus 2 100101-01-1-p02
Cluster 2: Allocated (Meta)
\end{verbatim}


\subsection{\emph{cluster extract}}

Prints the contents of the cluster.

\begin{verbatim}
 RVT  > cluster extract 3 100101-01-1-p02                                     
--------------------------------------------
/usr/bin/blkcat -o 0000000001 /media/morgue/imagenes/100101-espejismo/100101-01-1.dd 3 | 
\end{verbatim}


\subsection{\emph{cluster generateindex}}

Creates sort of an index for quick cluster-to-inode resolution. Required for performing searches. This is one of the few commands that writes files on the morgue.

See \emph{script search} (página \pageref{script:search} module for more information.


\subsection{\emph{cluster toinode}}

Prints all the inodes associated with a cluster.

\begin{verbatim}
 RVT  > cluster toinode 2 100101-01-1-p02                                     

inodes:

159700
\end{verbatim}


\section{Command: \emph{script}}

Modular scripts. Each module is explained in a separated chapter (see Chapter \ref{sec:modules} for further information)


\section{Command: \emph{set}}

Sets RVT Shell configuration information.

\subsection{\emph{set level}}

Sets the work level to a specific case, device, disk or partition. When the work level is stablished, is notified at the prompt and there is no need of indicate it as argument at the commands.

\begin{verbatim}

 RVT  > images partition table 100101-01-1                                    

        03:     6 MB    Linux (0x83)
        02:     5 MB    Linux (0x83)

 RVT  > 
 RVT  > images partition table


I don't know what is this

 RVT  > set level 100101-01-1                                                 

 new format: disk
 RVT 100101-01-1 > images partition table                                     

        03:     6 MB    Linux (0x83)
        02:     5 MB    Linux (0x83)

 RVT 100101-01-1 > 
\end{verbatim}












\chapter{Script modules} \label{sec:modules}

Scripts modules are forensic software components that perform specific forensic tasks on the stored images and information.

Each module has particular objectives, methods, arguments and results, the documentation of which is detailed in the following sections.





\section{\emph{script files}}

Module for performing operations on mounted images.

\subsection{\emph{script files allocfiles}} 

Creates a file with a list of all the allocated files of the image.

\begin{verbatim}
files allocfiles <disk>
\end{verbatim}

It creates a file in the morgue with this path and name:

\begin{verbatim}
<morguepath>/<case>/<disk>/output/info/alloc_files.txt
\end{verbatim}

with a list of allocated files and folders.





\section{\emph{script search}} \label{script:search}

Module for performing searches. See the Appendix \ref{anx:searches} for a step-by-step tutorial.


\subsection{\emph{script search clusterlist}}

Builds a list of clusters and file paths that matches a previous search.

\begin{verbatim}
script search clusterlist <search file> <image>
\end{verbatim}

It creates several files in the morgue with this path and name:

\begin{verbatim}
<morguepath>/<case>/<disk>/output/searches/cbusq_<search>-<partition>
\end{verbatim}

with this format:

\begin{verbatim}
<cluster>:<inode>:<allocation status>:<file path>
\end{verbatim}



\subsection{\emph{script search clusters}}

Extract the clusters matched in a previous search

\begin{verbatim}
script search clusters <search file> <image>
\end{verbatim}

It creates several files in the morgue with this path and name:

\begin{verbatim}
<morguepath>/<case>/<disk>/output/searches/ibusq_<search>-<partition>
\end{verbatim}

with this format:

\begin{verbatim}
---------------------------
<byte offset>:<cluster>:<allocation status>:  <line that matches search>

<cluster content>
\end{verbatim}


\subsection{\emph{script search file}}

Commands to manage search files:

\begin{itemize}
\item	script search file delete <name>:  deletes file
\item	script search file edit <name>:  creates and/or edit file
\item	script search file list:  lists search files created
\item	script search file show <name>:  prints file's contents
\end{itemize}

The files are created per case, and are stored at this path:

\begin{verbatim}
<morguepath>/<case>/searches_files
\end{verbatim}


\subsection{\emph{script search launch}}

Launches a search in an image. 

\begin{verbatim}
script search launch <search file> <image>
\end{verbatim}

It creates several files in the morgue with this path and name:

\begin{verbatim}
<morguepath>/<case>/<disk>/output/searches/busq_<search>
\end{verbatim}

with this format:

\begin{verbatim}
<strings file that matches>: <byte offset> <line that matches>
\end{verbatim}



\subsection{\emph{script search quickcount}}

Launch a quick search in a case or in an image with a count of the results.

\begin{verbatim}
script search quickcount <name:regular expression>  <image>
\end{verbatim}

It creates several files in the morgue with this path and name:

\begin{verbatim}
<morguepath>/<case>/<disk>/output/info/count_<name>
\end{verbatim}

with this format:

\begin{verbatim}
<times that the match appear>  <match>
\end{verbatim}

There are some quickcounts preconfigured:

\begin{itemize}
\item script search quickcount emails <image>: search emails addresses
\item script search quickcount accounts <image>: search bank accounts
\item script search quickcount ips <image>: search ip addresses
\item script search quickcount phones <image>: search phone numbers
\end{itemize}




\section{\emph{script strings}}

Module for creating and managing string files of images.

\subsection{\emph{script strings generate}} 

Generates strings for all partitions of a disk 

\begin{verbatim}
script strings generate <disk>
\end{verbatim}

It creates several files in the morgue with this path and name:

\begin{verbatim}
<morguepath>/<case>/<disk>/output/strings/strings-<case>-<partition>.asc
<morguepath>/<case>/<disk>/output/strings/strings-<case>-<partition>.uni
\end{verbatim}

Strings files of ASCII characters have \emph{.asc} extension.

Strings files of Unicode (UTF8) characters have \emph{.uni} extension.





\section{\emph{script timelines}}

Module for creating and managing timelines.

\subsection{\emph{script timelines generate}} 

Generates timelines for all partitions of a disk

\begin{verbatim}
script timelines generate <disk>
\end{verbatim}

It creates several files in the morgue with this path and name:

\begin{verbatim}
itimeline-<partition>.csv
itimeline-day-<partition>.sum
timeline.csv
timeline.txt
timeline-day.sum
timeline-hour.sum
\end{verbatim}

Where \emph{i}timelines are calculated based on inodes, and timelines, with file names. \emph{day} and \emph{hour} sums are line counts grouped by days and hours.




\section{\emph{script webmail}}

Module for detecting and managing webmail.

\subsection{\emph{script webmail detection}} 

Tries to detect certain types of webmail traces in the disk

\begin{verbatim}
script webmail detection <disk>
\end{verbatim}

It creates a file in the morgue with this path and name:

\begin{verbatim}
<morguepath>/<case>/<disk>/output/info/webmails.txt
\end{verbatim}

This file \emph{only} says if traces of several types of webmails are found. In this process, some searches have been launched, so further investigation can be performed on these searches.


\end{document}


